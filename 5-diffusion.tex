%!TeX root = 5-diffusion.tex
\documentclass[main]{subfiles}

\begin{document}

\chapter{Xenon and krypton transport properties}
\vspace*{-1\baselineskip}


In separation processes, diffusion can either be the main performance metric or a neglected secondary parameter. There are actually two different use cases for separation using nanoporous materials: the adsorption-based separation that is mainly a thermodynamic process and the nanoporous separation membranes that relies on the kinetic properties. In a membrane-based process, the gas is sieved through a membrane material that blocks some molecules (e.g. Xe) and let other molecules (e.g.) diffuse freely. The performance of the separation is measured with the ratio of the diffusion coefficients instead of the thermodynamic selectivity we defined in chapter 2. The process of interest is, however, the adsorption-based separation performed industrially by pressure and/or temperature swing adsorption, and even if the thermodynamic selectivity is the main performance metric, the kinetic performances can improve our understanding of the adsorption process. For instance, in breakthrough experiments (a lab equivalent of a pressure swing adsorption) used to characterize the comparative adsorption performances of a gas mixture, the shape of the curve can be explained by diffusion processes. The goal of this chapter is to explore this often neglected diffusion parameter in an adsorption-based Xe/Kr separation process.

\begin{figure}[ht]
  \centering
    \includegraphics[width=0.95\textwidth]{figures/5-diffusion/Diffusion.pdf}
    \caption{Illustration of the comparative role of the thermodynamic and transport properties for Xe/Kr separation in nanoporous materials. From the transport dominated process of membrane separation to the thermodynamically equilibrated separation processes in the nanopores, different more nuanced cases could emerge where the diffusion imposes kinetic limitations. }
    \label{fgr:intro_diffusion}
\end{figure}

\section{Modeling the Diffusion Process}

% From the macroscopic brownian motion modeled by the
% Fick's law
Since the pollen motion observations of the botannist Brown in 1826, scientists have observed and studied the seemingly erratic movement of particles in a static bulk medium. Later, Fick proposed a macroscopic model of this so-called brownian motion by introduced the coefficient $D$ of a diffusion equation~\ref{eq:fick} (1D) based on experimental measures of the concentration $\phi$.\autocite{Fick_1855} According to this law (valid only for independent particles), at the macroscopic level, the particles tend to move from the most concentrated area of the bulk to the less concentrated one. 

\begin{equation}\label{eq:fick}
  \frac{\partial \phi}{\partial t} = D_x \frac{\partial^2 \phi}{{\partial x}^2}
\end{equation}

To better understand the brownian motion of suspended particles on a liquid Einstein derived a microscopic model of the diffusion motion based on the molecular-kinetic theory of heat on the miraculous year of 1905.\autocite{einstein1905motion} To determine the so-called self-diffusion coefficient, he followed the motion of a particle assumed independent from other particles and time steps large enough to consider mutually independent two consecutive motions.  By using the particle distribution of $N$ independent diffusing particle, he redefined the diffusion coefficient as a function of the mean squared displacement (MSD) of a particle. In a tridimensional space, we have the following Einstein relation:
\begin{equation}\label{eq:einstein}
  \langle {r(t)}^2 \rangle = 2dD\e{diff}t
\end{equation}
where  $d$ is the dimension of the space in which the particle diffuses ($d=3$ in a volume) and $r(t)$ is the displacement of a particle from the time $0$ to $t$. The brackets represent the average over all independent trajectories (different particles and different time origins). This equation can be generalized to the diffusion of an adsorbate in the adsorbed phase, which describes how easy it is for a particle to move inside a nanoporous material. A low diffusion coefficient means a limited access to the pores of the structure as illustrated on Figure~\ref{fgr:intro_diffusion}.

Using molecular simulations of the adsorbate displacements, we will try to model the diffusion coefficient of xenon and krypton inside a nanoporous material. Although other approaches like the Green-Kubo equation exist the relatively less complex Einstein law is prefered for self-diffusion calculations, as shown by the following comparative study~\cite{Maginn_2020}. In this section, we will focus on the different simulation techniques that can be used to evaluate the diffusion in high-throughput screenings. We will try to present different ways of accessing the MSD of a diffusing particles, by beginning from the most straight-forward molecular dynamics to faster methodologies more suitable in screenings such as machine learned surrogate models and kinetic monte carlo simulations.


\subsection{Molecular dynamics}

Molecular dynamics are used to simulate the motion of molecules in a given system. It is usually used to calculate thermodynamic averagings.\todo{give some examples} Here, we are going to focus on the calculation of diffusion coefficients of monoatomic molecules.

\subsubsection{Simulation details}

Molecular dynamics (MD) aims at describing the motion of particles subjected to the forces of the surrounding particles. It can therefore be seen as a complex integration of the Newton's law of motion. A particle $i$ of position $\mathbf{r}_i$ and mass $m_i$ subjected to a force $\mathbf{F}_i$ resulting of the cumulated interactions with its surrounding is accelerated according to this equation:
\begin{equation}\label{eq:newton}
  m_i\frac{\dd^2 \mathbf{r}_i}{{\dd t}^2} = \mathbf{F}_i
\end{equation}

In a classical modeling, the forces are determined using the well-named forcefield that was previously introduced in the chapter 2. Back there, we only considered intermolecular interactions simply modeled by the Lennard-Jones (LJ) interaction potential between atom pairs, which is also what we will use in this section (of course, it is not the only way of defining a forcefield but just a simplification). Using the LJ potentials $U\ex{LJ}$ (defined in equation~\ref{eq:LJ}), we can derive a vectorial force $\mathbf{f}_{ij}$ between two atoms $i$ and $j$.
\begin{equation}
  \mathbf{f}_{ij} = - \left.{\dfrac{\dd U\ex{LJ}_{ij}}{\dd r}}\right\rvert_{r=r_{ij}} \frac{\mathbf{r}_{ij}}{r_{ij}} = 24\epsilon_{ij}  \left(2{\left(\dfrac{\sigma_{ij}}{r_{ij}}\right)}^{12} - {\left(\dfrac{\sigma_{ij}}{r_{ij}}\right)}^{6}\right) \frac{\mathbf{r}_{ij}}{r_{ij}^2}
\end{equation}
where $\epsilon_{ij}$ and $\sigma_{ij}$ are the LJ parameters of the pair of atoms $ij$. And the resulting force is simply the sum of the forces $\mathbf{F}_i=\sum_{j}\mathbf{f}_{ij}$ exerted by the surrounding atoms $j$. To reduce the computation time required, molecular simulations only consider the atoms within a given cutoff radius. 

Now that we defined the force $\mathbf{F}_i$, we can put a molecule in motion by integrating the equation~\ref{eq:newton} from a time $t$ to a time $t+\delta t$. There are different methods to integrate equation of motion such as the Euler or velocity-Verlet scheme presented in the book of Frenkel et al.\autocite{frenkel2001md} Here, we will focus on the \emph{leap frop} integration implemented in the RASPA\autocite{dubbeldam2016} software that we used for our simulations. The position $\mathbf{r}_i$ and the velocity $\dot{\mathbf{r}}_i$ are between each time step $\delta t$ using the following equations:
\begin{equation}\label{eq:frogleap_integration}
  \begin{split}
    \dot{\mathbf{r}}_i\left(t+\tfrac{1}{2}\delta t\right) & = \dot{\mathbf{r}}_i\left(t-\tfrac{1}{2}\delta t\right) + \tfrac{1}{m_i}\mathbf{f}_i \\
    \mathbf{r}_i\left(t+\delta t\right) & = \mathbf{r}_i\left(t\right) + \dot{\mathbf{r}}_i\left(t+\tfrac{1}{2}\delta t\right)\delta t
  \end{split}
\end{equation}
From the initial conditions $(\mathbf{r}_i(0),\dot{\mathbf{r}}_i(0.5\delta t))$, we can translate the center of mass of the molecule $i$ to any position $\mathbf{r}_i(t_n=n*\delta t)$. We will skip the rotation step required for polyatomic molecules since we are restricting the study on the monoatomic noble gas. The different positions ${\left\{\left(t_n,\mathbf{r}_i(t_n)\right)\right\}}_{n=0,\ldots,N\e{tot}}$ constitute the total trajectory of the MD simulation (to simplify I do not mention the velocities). It is possible to use this total trajectory to derive an average of MSD that could be analysed to calculate the diffusion coefficient.

\subsubsection{Diffusivity calculation using an MD trajectory}

I used the MSD sampling technique implemented in RASPA\autocite{dubbeldam2016} that was presented in an article~\cite{Dubbeldam_2009} by a few authors of the adsorption simulation software. The approach is based on a modified approach of the order-$\mathbf{n}$ algorithm described in the book~\cite{frenkel2001msd} of Frenkel and Smit. Therefore, I will focus on the so-called multiple window algorithm used to calculate the diffusion coefficients of xenon and krypton in this chapter. 

To understand it, I will start by explaining what a window algorithm would do and how it generalizes to the multiple window algorithm we are interested in.
First, let us consider a single MD trajectory of duration $t\e{tot}=N\e{tot}\delta t$. This trajectory can be used to generate displacement of any size $\tau$. Naively, we can start by taking ${\lVert\mathbf{r}_i(\tau)-\mathbf{r}_i(0)\rVert}^2$ as a square displacement of a sub-trajectory $\mathcal{T}(0\rightarrow\tau)$ of duration $\tau$. 
However, it is not enough to make a statistically meaningful average of the MSD as described in the Einstein equation\ref{eq:einstein}. Using the hypothesis of independence between two movements of the same particle separated by a time $\delta t$ used in Einstein's paper~\cite{einstein1905motion}, a shift of the origin of time by $\delta t$ would generate another trajectory. We can repeat this process $i$ times while $\tau + i\delta t\leq t\e{tot}$. This would be very accurate, but also very inefficient in the case where $\tau \gg \delta t$ since two consecutive sub-trajectories $\mathcal{T}(i\delta t\rightarrow\tau+i\delta t)$ and $\mathcal{T}((i+1)\delta t\rightarrow\tau+(i+1)\delta t)$ would be very similar. 

To efficiently sample the trajectory into sub-trajectories that are independent we can use a sampling time step of $\delta \tau\lesssim\tau$ chosen to be in the same order of magnitude as $\tau$. To do so, the window approach would first define a value $\delta \tau$ and generate $N_{\tau} = \lfloor(t\e{t ot} -\tau)/ \delta\tau \rfloor$ different sub-trajectories $\left\{\mathcal{T}(0\rightarrow\tau), \mathcal{T}(\delta\tau\rightarrow\tau + \delta\tau), \ldots, \mathcal{T}(N_{\tau}\delta\tau\rightarrow\tau + N_{\tau}\delta\tau)\right\}$ of duration $\tau=k\delta\tau$, where $k$ is an integer between $1$ and $K$ that defines the time window we want to sample. By doing so, we get the MSD $\langle {r(\tau)}^2 \rangle$ for duration values $\tau$ equal to $\delta\tau, \ldots, K\delta\tau$. The relation $\langle {r(\tau)}^2 \rangle$ can then be fitted to the equation~\ref{eq:einstein} to obtain the diffusion coefficient if the relation is linear. The trajectory generation of the window approach is illustrated on the Figure~\ref{fgr:window_msd} for a decomposition into sub-trajectories of a duration $\tau=3\delta\tau$ shifted by $\delta\tau$.

\begin{figure}[ht]
  \centering
    \includegraphics[width=0.95\textwidth]{figures/5-diffusion/diffusion_averaging.pdf}
    \caption{Illustration of the generation of trajectories of size $\tau$ by shifting the origins of multiple durations $\delta\tau$. }\label{fgr:window_msd}
\end{figure}

The major drawback of this method is that we need to define a timescale $\left\{\delta\tau, \ldots, K\delta\tau\right\}$ beforehand. In order to be able to access the different timescales in a single simulation, we can perform a multiple window algorithm developed by Dubbeldam et al.\ and used in the RASPA software to compute mean squared displacements (MSD) in a molecular dynamics simulation.

The different time windows are defined in a recursive way using the default parameters of RASPA. The first time window is defined by $K=25$ displacements of duration $\delta t, 2\delta t, \ldots,K\delta t$ with a shift of $\delta t$ (the default shift value $\delta t$ of the first window can be changed with the parameter SampleMSDEvery). The second window is now based on a sampling duration $\delta \tau_1 = K\delta t$ and the sub-trajectories will have durations of $\left(\tau_1^{(1)} = \delta\tau_1\right),\ldots,\left(\tau_1^{(k)} = K\delta\tau_1\right)$. If we repeat the process recursively until no window can be generated anymore, and the $i$\e{th} window would have a sampling duraction of $\delta \tau_i = K^i\delta t$ and sub-trajectories durations of $\left(\tau_i^{(1)} = \delta\tau_i\right),\ldots,\left(\tau_i^{(k)} = K\delta\tau_i\right)$. The time scale $\delta \tau_i = K^i\delta t$ we sample follows a geometrical progression and very different time scales can be accessed using this method in order to find the time scale corresponding to the diffusion regime (linear relation between the MSD and the duration of the sub-trajectories used in the averaging). For example on Figure~\ref{fgr:MSD_init}, we can see the different timescales and the exponent value $b$ of a fit to a function of type $x \mapsto ax^b$ for the different time windows --- values of $b$ near $1$ can be associated to a diffusion regime. The determination of the diffusion coefficient is now reduced to a simple fitting problem that will be explained in more details in the presentation of the diffusion coefficient screening in section~\ref{sct:md_screening}.

This methodology can then be used replicated to thousands of structures to characterize the diffusion properties of these materials. Several screenings have already been carried out in the literature as presented in the chapter 1 in the section dedicated to transport property screenings. We will now dive a little deeper in the prediction of these quantities using machine learning.

\subsubsection{ML modeling}

In a very recent study, Daglar et al.\ used an ML model to predict the diffusion coefficient of a 100 thousand hypothetical MOFs using the data for about 5000 CoRE MOF structures.\autocite{Daglar_2022} Along with very standard geometrical descriptors, they used chemical composition descriptors and the heat of adsorption as the input features of their machine learning model to predict the diffusion coefficients of \ce{H2}, \ce{CH4}, \ce{N2} and \ce{He} in the different MOF materials of CoRE MOF 2019 (training) and of hMOF (testing). The combination of kinetic data with thermodynamic data for the characterization of MOf materials is a very interesting approach. However, the major drawback of most of the approaches in the literature is the lack of structure--property relationship to understand the microscopic origins of the diffusion coefficient values.

Simarly to what we have done on the thermodynamic screening (chapter 2--4), in our approach to tranport property screening, we will also start by drawing structure--property relationships between the diffusion coefficient and the geometrical descriptors of the MOF structures. And in an attempt to have a deeper understanding of the diffusion process, we will try to evaluate the diffusion activation energy using energy grid-based methods described in the literature. All these techniques aim at better predicting the diffusion coefficients either in a direct calculation or in an ML surrogate model. To achieve that, we will start by introducing the kinetic Monte Carlo approach that is less accurate than the MD approach but is indeed much more efficient.

\subsection{Lattice kinetic Monte Carlo}

The lattice kinetic Monte Carlo method relies on a pre-defined lattice of stable points corresponding to adsorption sites. Each site connected to another if there is a diffusion path (narrow channel) that connects them. To calculate the probability of transition from one site to another, we need to define a transition state in the narraow channel which correspond to the highest energy point on the minimal energy diffusion path (the saddle point). The probability of transition would therefore be defined with regard to the energy barrier to overcome in order to cross the channel. Once the lattice defined, we only need to propagate an adsorbate from one site to another using the different transition probabilities, which gives a coarse-grained trajectory compared to the one obtained in a MD simulation, but is perfectly usable to compute the MSD and calculate a diffusion coefficient.


\begin{figure}[ht]
  \centering
    \includegraphics[width=0.95\textwidth]{figures/5-diffusion/kinetic_MC.pdf}
    \caption{Illustration of the core principle of lattice kinetic Monte Carlo in a periodic system. The particle moves inside the periodic system according to the transition probabilities to move from one site to another. The transition rates are determined using the activation energy needed to move to the transition state between the two stable adsorption sites as shown in equations~\ref{eq:trans_rate_path} and~\ref{eq:trans_rate}. }\label{fgr:kMC_principle}
\end{figure}

\subsubsection{Transition state theory for diffusion}

The transition state theory is usually used in chemistry to explain the kinetics of a reaction. To do so it compares the energy of the reactants before reacting and the one of the transition state to calculate the rate of the reaction. This reaction rate along a reaction path is proportional to the ratio of the Boltzmann factor at the transition state and the integration of the Boltzmann factors along the reaction path.

This definition can be directly transposed to the case of a diffusion path instead of a reaction path, and we can define the diffusion rate $\mathcal{R}_{0\rightarrow 1}$ from site $0$ to $1$ as follows:
\begin{equation}\label{eq:trans_rate_path}
  \mathcal{R}_{0\rightarrow 1} = \kappa\sqrt{\frac{k\e{B}T}{2\pi m}}\frac{e^{-\beta E(\mathbf{r}\ex{TS})/k\e{B}T}}{\int\e{path}e^{-\beta E(\mathbf{r})/k\e{B}T}\dd\mathbf{r}}
\end{equation}
where $\kappa$ is the Bennet-Chandler dynamic correction factor,\autocite{BENNETT1977} and  $\kappa=\tfrac{1}{2}$ if it is equiprobable to reach both sites from the transition state. This approach calls for the determination of the optimal diffusion path before carrying out the determination fo the diffusion rate. This is for instance the approach adopted by Wang et al.\ in their screening for noble gas separation, where they determined the minimal energy path before calculating diffusion rates.\autocite{Wang_2022}

Another approach is to determine a surface of transition through which the adsorbate would pass to diffuse along the channel of the material. This approach calls for the determination of the surface of transition only, but it relies on another definition of the transition rate $\mathcal{R}_{0\rightarrow 1}$:
\begin{equation}\label{eq:trans_rate}
  \mathcal{R}_{0\rightarrow 1} = \kappa\sqrt{\frac{k\e{B}T}{2\pi m}}\frac{\int_{\mathcal{S}(TS)}e^{-\beta E(\mathbf{r})/k\e{B}T}\dd\mathbf{r}}{\int_{\mathcal{V}(\text{S}_0)}e^{-\beta E(\mathbf{r})/k\e{B}T}\dd\mathbf{r}}
\end{equation}
where the integration for the transition state is done on a bottle-neck surface that a diffusing particle need to cross in order to go from the volume occupied by a site $0$ to the one occupied by $1$.

\subsubsection{Construction of the lattice}

In this section, I will focus on the second approach that relies on the determination of a transition state as a surface. This approach has been developed by Mace et al.\ and relies on detecting the merging points of bassin representing the adsorption sites. The algorithm developed is called TuTraST, which stand for Tunnels and Transition States, is a search algorithm that aims at finding the tunnels and the transition states within them.\autocite{Mace_2019} Once the adsorption sites, the transition states and the tunnels that connect them are found, the equation~\ref{eq:trans_rate} can be used to calculate the transition rates to move a particle from one point to another and hence producing a kinetic Monte Carlo simulation to determine the MSD and eventually the diffusion coefficients. 


\todo{Give the scheme here}

This approach is very promising since it is much more efficient than the MD simulation based techniques. with an implementation in Matlab, the code is already out-performing most of the MD simulations for diffsuion coefficient calculation, with a minimal cost in accuracy as shown in their screening of diffusion coefficients in zeolites. By using the C++ programming language, I tried to rewrite the algorithm for a more efficient search of the transition states. At this point of the development, I stopped at the determination of the diffusion activation energy which is independent of the detection of the transition state. And, in section~\ref{sct:algo_diff}, I will give the algorithmic detail for the determination of the diffusion activation energy in nanoporous mateirals using our in-house algorithm as well as the projected develoopment to achieve a faster lattice kinetic Monte Carlo simulation inspired by TuTraST.


\section{Screening of transport properties}\label{sct:md_screening}

To complete the thermodynamic screenings that we performed in the chapters 2--4, we also carried out a transport property screening. In this section, we will provide a description of the screening approach as well as the analysis of the diffusion coefficients compared with typical geometrical descriptors.

\subsection{Diffusion in a selective material}

Before going into the details of the screening study, we will present the approach adopted for the diffusion coefficient calculation using MSD values, on one example, SBMOF-1\autocite{Banerjee_2016}. This preliminary study will help us calibrate the time parameters (time step, maximum time) that will be used in the final screening study.

First, I ran a molecular dynamics simulation of 500 million steps (about 1--2 days of simulation) with a thousand initialization steps and 100 thousand equilibration steps to model a xenon diffusing in the KAXQIL\autocite{Banerjee2012} MOF at infinite dilution. To be at infinite dilution, we set the box size so that no interactions occur between the different adsorbates. And we can observe on the Figure~\ref{fgr:MSD_init}, the different time scales at which different transport phenomena occur. 

From \SI{1}{\fs} to \SI{1}{\ps}, there is a ballistic regime with a squared dependence of the mean squared displacement. For a particle of mass $m$, the MSD $\langle {r(\tau)}^2 \rangle$ in this regime follows a simple ballistic relation (length equals velocity multiplied by time):
\begin{equation}
  \langle {r(\tau)}^2 \rangle = v_m^2 \tau^2 = \frac{k\e{B}T}{2\pi m}\tau^2
\end{equation}
where $v_m$ is the average velocity of a particle that follows the Maxwell-Boltzmann distribution at temperature $T$. If we calculate the squared mean velocity $v_m^2$ using the standard Maxwell-Boltzmann relation, we get a value of \SI{3}{\square\m\per\square\second}, which is very close to the value of \SI{2.8}{\square\m\per\square\second} obtained by the fit shown right plot of Figure~\ref{fgr:MSD_init}. This first regime simply corresponds to the movement of the particles subjected to the thermal agitation and is of little interest for diffusion. 

\begin{figure}[ht]
  \centering
    \includegraphics[width=0.48\textwidth]{figures/5-diffusion/MSD_anomalous_diffusion.pdf}
    \includegraphics[width=0.48\textwidth]{figures/5-diffusion/MSD_Xe_KAXQIL_clean.pdf}
    \caption{Left: Different regimes that could be observed in an MSD plot as a function of time. The ballistic regime can be considered superdiffusional, the normal diffusion is simply a linear relation as described in the Einstein equation\ref{eq:einstein}, and the subdiffusion regime often occurs in obstructed media like nanoporous materials. The different regimes can be found on the right plot of an actual MSD calculated using the multiple window method. The fittings are done using a generic function $K_\alpha\tau^\alpha$ and the exponents $\alpha$ are given in the legend. }\label{fgr:MSD_init}
\end{figure}

Then, there is a transition from the ballistic regime to the pseudo-diffusional regime (the exponent does not reach $1$ yet) that we observe in cyan on the plot. Between \SI{1}{\ps} and \SI{100}{\ps}, there is a so-called subdiffusion regime, where the MSD has a power function of the time $\langle {r(\tau)}^2 \rangle=K_\alpha\tau^\alpha$ with an exponent inferior to $1$ as illustrated on the left plot of Figure~\ref{fgr:MSD_init}. This regime corresponds to the confinement of the xenon particle inside an adsorption pore, there is only thermal vibration occuring and no diffusion hopping is observed at this time scale. And diffusion seems to start occuring at the \SI{10}{\ns} time-scale. As we can see on the Figure~\ref{fgr:MSD_linear_init}, the MSD between \SI{0.01}{\ns} and \SI{0.4}{\ns} really represents a sub-diffusional regime due to the confinement imposed by the nanopores of KAXQIL, but at \SI{0.4}{\ns}--\SI{9}{\ns}, the MSD start to have an exponent closer to $1$ and a linear fit is possible although not perfect. Ideally, we would want to sample trajectories in the order of tens of nanoseconds, which is the next time-scale. But with an MD time step of \SI{1}{\fs}, this would mean multiplying the computation time by at least 5 (1--2 weeks for one MD simulation), which begins to be prohibitive. 

\begin{figure}[ht]
  \centering
\includegraphics[width=0.48\textwidth]{figures/5-diffusion/MSD_Xe_coeff_KAXQIL_clean_1.pdf}
\includegraphics[width=0.48\textwidth]{figures/5-diffusion/MSD_Xe_coeff_KAXQIL_clean_2.pdf}
\caption{ Plots of the MSD at the last two time scales considered on the Figure~\ref{fgr:MSD_init}. On the left, the time-scale between \SI{0.01}{\ns} and \SI{0.4}{\ns} is considered, the MSD is fitted by a power function with the same exponent as one determined earlier, and a linear fit is given to show the incompatibility with the diffusion equation. On the right, we have the same approach but for the time-scale between \SI{0.4}{\ns} and \SI{9}{\ns}. }\label{fgr:MSD_linear_init}
\end{figure}

If we use the right plot of Figure~\ref{fgr:MSD_linear_init} to fit a linear relation and deduct the diffusion coefficient, we can have an underestimated value of the diffusion coefficient of $2.24\times 10^{-8}$~\si{\square\cm\per\s} --- it is an underestimation because the MSD is rather concave, which reduces the slope in the fitting process. This value is already a good estimation of the diffusion coefficient considering the rather high exponent $\alpha$ value in the with regard to $K_\alpha\tau^\alpha$.

Since there is an element of randomness in the initial position of xenon (block pockets have been calculated for a \SI{1.5}{\angstrom}-radius probe), we need to measure the effet of running different MD simulations of the value of the diffusion coefficients. To measure this uncertainty across different MD simulations with different initial positions determined by with different random seed. In RASPA, the random seed is simply equal to the UNIX time upon launching the MD simulation. This ensures that a different random seed is given to the $10$ different MD simulations we launched using the exact same parameters as mentioned previously. We found that the average diffusion coefficient value equals $2.13\times 10^{-8}$~\si{\square\cm\per\s} and the standard deviation equals $0.37\times 10^{-8}$~\si{\square\cm\per\s}, which represents about {17\%} of the average value. We could estimate the uncertainty to about {17\%} on the diffusion coefficient for a rather low coefficient around $10^{-8}$~\si{\square\cm\per\s}, we could expect lower uncertainty for less obstructing materials. 

Now that we have more confidence on the method we are using, we will try to probe higher time-scales than the one accessible with an MD time step of \SI{1}{\fs}, because the diffusion regime seems to be occuring rather at the \SI{10}{\ns} time-scale. We ran a first calculation with 500 million step to confirm that we obtain a similar diffusion coefficient value. The time window between \SI{2}{\ns} and \SI{47}{\ns}, and the MSD are calculated from about 200 sampled trajectories, which gives rather correct values. We can obtain a more accurate diffusion coefficient of $2.6\times 10^{-8}$~\si{\square\cm\per\s}, which is very close to the one obtained in the previous approach. The value is slightly higher (which is expected) since the previous values was over-estimated. This approach is therefore consistent with the previous one.

To further back-up the use of a higher time integration step, we need to understand the origin of the value of \SI{1}{\fs}. This value is usually justified by the Nyquist-Shannon sampling theorem that implies the integration step to be at most half of the period of the fastest vibration within the system. If we take a C--H vibration, the maximum time step value is \SI{5}{\fs}, and to be on the safe side, a time step of $1$--$2$~\si{\fs} is chosen in most of the diffusion studies in nanoporous materials.\autocite{Bukowski_2021} But in our system of a xenon diffusing in a rigid environment, we actually don't have any vibrational limitations as described earlier. I think that the use of higher time steps in this situation can be an easy way to access higher time-scales; however, furhter studies need to be performed to be sure of the validity of the quantities we derive from these MD simulations. The value of \SI{5}{\fs} is on the higher side of what we usually see in MD simulations, but it can be justified by the rigidity of the framework and the adsorbate we consider. Even higher time steps could be tested, but to be sure of the results we stay at this reasonable middle ground of \SI{5}{\fs} for all our high-throughput screening of the transport properties.


\subsection{High-throughput screening of diffusion coefficients}

\subsubsection{Screening procedure}

In order to include the transport properties in our analysis, I performed MD calculations of a xenon or krypton at infinite dilution (no guest--guest interactions) for the 6,525 non-disordered most selective materials. For each of them, 500 million steps of MD were planned in the RASPA script in the calculation machines and 2 to 3 restarts were done on the slowest simulations so that every MSD data were obtained after 2--3 days. After this process, only 432 structures have finished the planned 500 million steps, but it does not mean that the MSD is not exploitable for the determination of the diffusion coefficients. 

To determine the diffusion coefficients we analyzed two time-scales (2--47\si{\ns} and 50--950~\si{\ns}) to fit the MSD with a linear function. We chose the linear fit with the highest determination coefficient (within $0$ and $1$) of both time-scales to output a value of diffusion coefficient. After this step, we removed structures with a determination coefficient lower than $0.9$ and used the 5,125 remaining structures to draw structure--diffusivity relationships --- these structures for which we have a good degree of confidence on the diffusion coefficient values will be comparatively studied against different geometrical and thermodynamic quantities in this section

This approach only probes the linear relations between the MSD and time to determine the self-diffusion at infinite dilution. We did not analyze the nature of the transport property (e.g. single-file diffusion\autocite{Lin_2005}) by comparing for example the exponent of a generalized formula $\text{MSD}(/tau) = K_\alpha\tau^\alpha$ with structural dscriptors. We did not study the more complex diffusion at higher loading values, which could be more accurately described by a collective diffusion coefficient instead of the self-diffusion coefficient. The goal of this study is to find materials that do not present a kinetic limitation as it is the case for KAXQIL\autocite{Banerjee2012} (xenon has a diffusion coefficient that is about ten thousand times lower in the material than outside).

\subsubsection{Structure--diffusivity relationships}

In this section, I will present the different relations the diffusion coefficient may or may not have with the geometrical descriptors. I decided to use a forcefield-dependent definition of the radii so that it better correlates with the results of the MD simulation that uses the UFF forcefield. The different geometrical descriptors are then calculated using Zeo++ and these radii to calculate the PLD, the largest sphere diameter D$_{if}$ along a free path, the surface area and the pore volume, as already explained in the chapter 2. To further justify the use of the UFF-based radii, the original paper~\cite{Hung_2021} showed the better correlation of the PLD to the diffusion constant, and we can also see it from the Figure~\ref{fgr:diff_pld}. The PLD calculated by the standard CCDC defined atom radii does not fit the diffusion coefficient as well as the UFF-defined PLD. As shown in a smaller scale in the article~\cite{Haldoupis_2010} (see Figure~\ref{fgr:Haldoupis_2010} in the chapter 1), there is a linear relation between the diffusion activation energy (logarithmic transform of the diffusion coefficient) and the PLD, and this linear relation is much more noisy for the PLD defined by the standard CCDC radii than for the UFF-based PLD.

Beyond the practical considerations on the geometrical descriptor calculations, the PLD explains the outlines of the variation of diffusion performance inside a nanoporous material. First there is a the linear relation previously highlighted, and then there is a sort of noisy plateau. In the first zone, the xenon is constrained by channels narrower than its kinetic radius. The wider the channel the higher the diffusion coefficient is, and this positive correlation persists until the channel is wider than about \SI{4.6}{\angstrom}. After this threshold, the diffusion coefficient is rather stable around $3\times 10^{-4}$~\si{\square\cm\per\s}, and the variations can only be explained by other phenomena such as the tortuosity inside nanopores or chemical nature of the surface of the nanopores. This value of diffusion coefficient can be interpreted as the diffusion coefficient of a ``free'' xenon that is less affected by the surrounding pore surface. The channels are large enough so that the xenon is only a little slowed down, for values of the PLD over \SI{5}{\angstrom}. These results are compatible with experimental data that measured the diffusion coefficient of xenon dissolved in water at different temperature conditions, and found a value of $10^{-5}$~\si{\square\cm\per\s} at \SI{303}{\kelvin},\autocite{Wise1968} which is consistent with values centered around $3\times 10^{-4}$~\si{\square\cm\per\s} at the plateau.

\begin{figure}[ht]
  \centering
    \includegraphics[width=0.48\textwidth]{figures/5-diffusion/D_log-diameter_ccdc_colored_s_+.pdf}
    \includegraphics[width=0.48\textwidth]{figures/5-diffusion/D_log-diameter_colored_s_+.pdf}
    \caption{Xenon diffusion coefficient at infinite dilution as a function of the pore limiting diameter (PLD). The diameter of the largest free sphere is defined using two different radius systems: the standard CCDC-based PLD (on the left), and the one defined using the UFF forcefield (on the right)\autocite{Hung_2021} --- as defined in the chapter 2. }\label{fgr:diff_pld}
\end{figure}

If we now analyze the channel dimensions (determined using Zeo++) that could partially inform us on the channel shape. As we can see on the Figure~\ref{fgr:scatter_diffusion_chandim}, the dispersion of the diffusion coefficients at the plateau is actually very hard to characterize using the channel dimension with bear eyes. 

\begin{figure}[ht]
  \centering
    \includegraphics[width=0.32\textwidth]{figures/5-diffusion/D_log-PLD_1D_chan.pdf}
    \includegraphics[width=0.32\textwidth]{figures/5-diffusion/D_log-PLD_2D_chan.pdf}
    \includegraphics[width=0.32\textwidth]{figures/5-diffusion/D_log-PLD_3D_chan.pdf}
    \caption{ Distributions of the base-10 logarithm of the diffusion coefficients of three different subsets of the screened structures. The first one (a) is composed of structures with a unidimensional channels, the second (b) bidimensional channels and the third one (c) tridimensional channels. }\label{fgr:scatter_diffusion_chandim}
\end{figure}

For this reason, I plotted on Figure~\ref{fgr:hist_diffusion_chandim} the distribution of diffusion coefficients that depends on the dimensionality of the channels within the framework. The distribution for structures containing 1D structures is much more heavy tail in terms of low diffusion coefficients. We can more easily find 1D structures with very low diffusion coefficients under $3/times 10^{-8}$~\si{\square\cm\per\s}. The vast majority of structures with tridimensional channels has a rather higher diffusion coefficient between $3\times 10^{-6}$~\si{\square\cm\per\s} and $10^{-4}$~\si{\square\cm\per\s} with almost no structures with lower diffusion coefficients. This is not as blatant for bi- and unidimensional channels, we can more easily find structures between $3/times 10^{-8}$~\si{\square\cm\per\s} and $3\times 10^{-6}$~\si{\square\cm\per\s}, even if thery are not that frequent. The dimensionality of the channels can therefore influence the values of diffusion coefficient but the relation is not as clear as for PLD.

\begin{figure}[ht]
  \centering
    \includegraphics[width=0.32\textwidth]{figures/5-diffusion/histogram_chan1D.pdf}
    \includegraphics[width=0.32\textwidth]{figures/5-diffusion/histogram_chan2D.pdf}
    \includegraphics[width=0.32\textwidth]{figures/5-diffusion/histogram_chan3D.pdf}
    \caption{ Distributions of diffusion coefficient of three different subsets of the screened structures. The first one (a) is composed of structures with a unidimensional channels, the second (b) bidimensional channels and the third one (c) tridimensional channels. }\label{fgr:hist_diffusion_chandim}
\end{figure}

Other geometrical properties of the material that can influence the diffusion is the void fraction and surface area. The low diffusion coefficients usually happen in materials with small pore volumes below $0.6$ as we can see on the Figure~\ref{fgr:diff_sa_vf}. However, it is diffuscult to draw a relationship between the voide fraction and the diffusion coefficient. The only relation is that materials with void fraction higher than $0.6$ have a diffsion coefficient over $3\times 10^{-6}$~\si{\square\cm\per\s}. This phenomenon is certainly du to the correlation between the PLD and the void fraction. Large PLD values are usually associated with large values of the void fraction. On the other hand, the accessible surface area for a probe of size $1.2$~\si{\angstrom} does not seem to influence the diffusion coefficient at all. 

\begin{figure}[ht]
  \centering
    \includegraphics[width=0.48\textwidth]{figures/5-diffusion/D_log-vf_2_s_+.pdf}
    \includegraphics[width=0.48\textwidth]{figures/5-diffusion/D_log-sa_12_s_+.pdf}
    \caption{Xenon diffusion coefficient at infinite dilution as a function of the accessible surface area (left) and the void fraction (right). }\label{fgr:diff_sa_vf}
\end{figure}

Framework density and molar mass are immediate characterictics of the structure since they do not require complicated simulations to obtain. The relation to the diffusion is, however, not that clear as we can see of the Figure~\ref{fgr:diff_density_mass}. We can probably say that low values of the density favors high values of the diffusion coefficeint, this can be understtod by the logical relation between low density and high porosity. Another relation to the diffusion coefficient concerns the higher probability to find high diffusion coefficients as materials have a higher molar mass. This last assertion is harder to justify using simple geometrical reasoning. 

\begin{figure}[ht]
  \centering
    \includegraphics[width=0.48\textwidth]{figures/5-diffusion/D_log-density_s_+.pdf}
    \includegraphics[width=0.48\textwidth]{figures/5-diffusion/D_log-mass_s_+.pdf}
    \caption{Xenon diffusion coefficient at infinite dilution as a function of the density (left) and the mass (right) of the frameworks. }\label{fgr:diff_density_mass}
\end{figure}

The largest sphere diameter D$_{if}$ along a free diffusion path has a similar relation to the diffusion coefficient, but the correlations are much more noisy as we can see on the left plot of the Figure~\ref{fgr:diff_H_lcd}.  This can be explained by the fact that D$_{if}$ is always superior or equal to the pore limiting diameter  D$_{f}$ by definition. When both are equal it is equivalent to the Figure~\ref{fgr:diff_pld} with a linear relation and plateau, but if it is higher it creates the sort of noise that we see on the the left plot of the Figure~\ref{fgr:diff_H_lcd}. 

\begin{figure}[ht]
  \centering
    \includegraphics[width=0.48\textwidth]{figures/5-diffusion/D_log-lcd_s_+.pdf}
    \includegraphics[width=0.48\textwidth]{figures/5-diffusion/D_log-H_Xe_s_+.pdf}
    \caption{Xenon diffusion coefficient at infinite dilution as a function of the largest sphere diameter D$_{if}$ along a free diffusion path (left) and the xenon adsorption enthalpy (right). }\label{fgr:diff_H_lcd}
\end{figure}

The last comparison is made with a thermodynamic quantity, the xenon adsorption enthalpy $\Delta\e{ads}\ex{Xe}H$. There is no relation between diffusion coefficient and the xenon adsorption enthalpy, which is good thing because it means that any configurations are possible. A material can both have a high diffusion coefficient and a high xenon adsorption affinity (very negative values of enthalpy), which is the best configuration for adsorption at infinite dilution. However, it also means that we need to test the diffusivity when the material has a good affinity in order to optimize both properties. This approach will constitue the core discussion around the optimization of both the Xe/Kr selectivity and the diffusion coefficients of Xe and Kr.  

\subsection{A trade-off between the selectivity and the diffusion}

In this section, I will analyze the screening of the diffusion and selectivity properties calculated for xenon and krypton to identify interesting materials that presents both a good Xe/Kr selectivity and a good Xe/Kr diffusion coefficient ratio. To do so we also performed a diffusion coefficient screening for krypton, and we end up with 4,816 structures that has good determination coefficient for both linear fits of xenon and krypton MSD. These structures are then tested to find materials with a good balance between thermodynamic and kinetic properties for the xenon/krypton separation.

\subsubsection{Screening trade-off between }\label{sct:diff_screen}

I will start by comparing the xenon/krypton selectivity at infinite dilution with the xenon and krypton diffusion coefficients. First, we can say that a highly selective materials can have a decent diffusion coefficient, the diffusion limitation observed on the KAXQIL structure is therefore not inevitable, which is a good news. On the left plot of the Figure~\ref{fgr:diff_s0_lcd}, we can clearly see that all configurations are possible: high selectivity (over $40$) with high diffusion coefficient (over $10^{-6}$~\si{\square\cm\per\s}) and high selectivity with low diffusivity. The krypton coefficients seem to be rather stable between $10^{-6}$~\si{\square\cm\per\s} and $10^{-3}$~\si{\square\cm\per\s}. It means that it is not the main leverage to increase the diffusion selectivity since very low values of kryton diffusion coefficients do not appear for highly selective materials. To improve the tunnel vision we had on the thermodynamic selectivity, we will study a transport related selectivity metric to find highly selective materials that do not show significant transport limitations.

\begin{figure}[ht]
  \centering
    \includegraphics[width=0.48\textwidth]{figures/5-diffusion/D_xe-s0-lcd.pdf}
    \includegraphics[width=0.48\textwidth]{figures/5-diffusion/D_kr-s0-lcd.pdf}
    \caption{}\label{fgr:diff_s0_lcd}
\end{figure}

To take account of the transport properties in a separation process, we generally use the ratio of the diffusion coefficients or the diffusion selectivity as a performance metric. For xenon and krypton the diffusion selectivity can be defined as follows:\autocite{Krishna_2010}
\begin{equation}
  s\ex{Xe/Kr}\e{diff} = \frac{D\ex{Xe}}{D\ex{Kr}}
\end{equation}
If we want to look at both the transport and thermodynamic effects, we can combine the thermodynamic adsorption selectivity defined in chapter 2 (equations~\ref{eq:selec_0} and~\ref{eq:selec_0}) and the diffusion selectivity to define the membrane selectivity (used to characterize membranes). This membrane selectivity can also be called a perm-selectivity because it corresponds to the ratio of the permeabilities of the components of the binary mixture we want to separate. The xenon/krypton permselectivity $s\ex{Xe/Kr}\e{perm}$ can therefore be defined as follows:
\begin{equation}\label{eq:membrane_selec}
  s\ex{Xe/Kr}\e{perm} = s\ex{Xe/Kr}\e{diff} \times s\ex{Xe/Kr}\e{ads}
\end{equation}
where $s\ex{Xe/Kr}\e{ads}$ corresponds to the adsorption selectivity used throughout the previous chapters (at infinite dilution or higher pressure). 

\begin{figure}[ht]
  \centering
    \includegraphics[width=0.48\textwidth]{figures/5-diffusion/diff_D_xekr-s0-lcd.pdf}
    \includegraphics[width=0.48\textwidth]{figures/5-diffusion/diff_D_xekr-s0-chandim.pdf}
    \caption{}\label{fgr:perm_selec0}
\end{figure}

Using both $s\ex{Xe/Kr}\e{diff}$ and $s\ex{Xe/Kr}\e{ads}$ at different pressure conditions, we will try to find materials that exhibit a rather high selectivity with a high diffusion selectivity. The plots of the Figure~\ref{fgr:perm_selec0} shows that 48 structures have a selectivity over $40$ with a rather good diffusion selectivity over $0.1$. These structures have rather high values of pore size represented by the largest included sphere along a free diffusion path $D_{if}$ as we can see on the left plot of the Figure~\ref{fgr:perm_selec0}. And these rather large pores are associated with structures with different dimensionalities.
One remarkable structure among them has a very high diffusion selectivity (over $15$, grey point on the upper right side of the left plot of Figure~\ref{fgr:perm_selec0}) coupled with a high adsorption selectivity at infinite dilution. This structure, with a CSD code ADOGEH\cite{Peikert_2012}, has a tridimensional channel framework with large pores and rather narrow connecting channels. The high adsorption selectivity at infinite dilution is however not conserved at ambient pressure as we can see on the Figure~\ref{fgr:perm_selec2080}  (see Table~\ref{table:diff}).

If we look into the details of these structures, we can see that they combine large pores with smaller pores so that the diffusion is not obstructed and the selectivity is high in more confined spaces. We saw that these type of materials with different sizes of pore could cause a selectivity drop at higher pressure, because the larger pores are less selective and are accessed when the gas pressure is increased. This phenomenon is observed when we compare the plots to the ones of Figure~\ref{fgr:perm_selec2080}.

\begin{figure}[ht]
  \centering
    \includegraphics[width=0.48\textwidth]{figures/5-diffusion/diff_D_xekr-s2080-lcd.pdf}
    \includegraphics[width=0.48\textwidth]{figures/5-diffusion/diff_D_xekr-s2080-chandim.pdf}
    \caption{}\label{fgr:perm_selec2080}
\end{figure}


We can see that at higher pressure, some materials experience a shift toward lower selectivity values. And, only 2 structures have an ambient-pressure selectivity over $40$: the MOFs with the following CSD code XUNSOQ\autocite{Abrahams_2014} and GUMDEZ\autocite{Yin_2014}  (see Table~\ref{table:diff}). Most of these materials actually have a high value of cavity size, only structures with LCD near \SI{6}{\angstrom} are left in this area of the plot. We can also note that the dimension of the channel is equal to one too, which starts to brush a picture of the interesting materials. These materials are constituted of unidimensional channels with a small pore sizes so that the selectivity is conserved even at higher pressure conditions. If we broaden the scope to the structures with a selectivity higher than $30$ instead of $40$, we now have 38 structures that also have similar features with rather low pore sizes and low channel dimensionality. Some of these structures have more or less kept their selectivity such as QOZDOY\autocite{Zhang_2001} and wasn't detected on the pre-screening on the low-pressure selectivity (Figure~\ref{fgr:perm_selec0}), but other structures, such as the MOF MISQIQ\autocite{Tong_2013}, actually dropped significantly in selectivity values from infinite dilution to ambient pressure (see Table~\ref{table:diff}). 

When considering the ambient-pressure selectivity, the large majority of highly selective materials actually have a rather low diffusion selectivity (lower than $0.1$), as we can see on the Figure~\ref{fgr:perm_selec2080} (this was not the case for the low-pressure selectivity). This result suggest that a trade-off between adsorption selectivity and diffusion selectivity is actually needed. In my screening approach, I decided to lower the adsorption diffusivity to values around $40$ to allow for higher diffusion selectivity values, because up until now standard screenings in the literature\autocite{Simon_2015,Chung_2019} and in my published work~\ref{Ren_2021} only mazimized the adsorption selectivity --- this equivalent to working on the lower right side of the plots of Figure~\ref{fgr:perm_selec2080}. To improve the former approach, we included a kinetic constraint in our screening. Another approach would be to optimize the perm-selectivity also known as the membrane selectivity (equation~\ref{eq:membrane_selec}), but it would be the solution to another application, the membrane separation, which is broadly studied in the literature\autocite{Anderson_2017,Wang_2022}. From the screening we presented here, we would rather like to find thermodynamically selective materials that are not limited by diffusion, and some interesting identified materials will be further studied in the following subsection. 


\subsubsection{Identification of interesting materials}

By crossing the transport data with the thermodynamic one, we can optimize the Xe/Kr adsorption selectivity under some constraint on the diffusion selectivity so that it is in an acceptable range (over $0.1$). The structures of the 65 structures with a low-pressure Xe/Kr selectivity higher than $40$ or with an ambient-pressure Xe/Kr selectivity higher than $30$ have been manually visualized and quickly analyzed, and different materials have been hand-picked for further analysis due to some special characterictics. We discarded some materials that have different type of channels that can artificially have a high diffusion coefficient. This phenomenon is due to the randomness of the initial condition, for example, when the xenon diffuses in a wider channel while the krypton diffuses in the narrower channel, the diffusion selectivity will inevitability be artificially higher. For example, the MOF with a CSD code OQESAF\autocite{Xie_2011} was concerned by this phenomenon as we can see on Figure~\ref{fgr:OQESAF} it clearly has different diffusion coefficeint values depending on the channel considered (a Henry coefficient weighted average need to be performed in this case). Other materials have a moderately high diffusion selectivity ($\lesssim 1$), they are usually composed of a unidimensional channel that allows a rather free diffusion of a xenon (higher than \SI{4.6}{\angstrom}) with different cavity sizes. Different factors seem to influence the diffusion coefficients, the values of the channel size and the pore size could explain the values of the diffusion and adsorption selectivity, but more interestingly the shape of the channel composed of cavities connected by narrower walls is also very important. Depending on the tortuosity of the layout and the relative difference between the cavities and the connecting channels, diffusion properties can be very different. 

\begin{figure}[ht]
  \centering
    \includegraphics[width=0.48\textwidth]{figures/5-diffusion/viz/OQESAF.jpg}
    \caption{Wrongly attributed to high diffusion selectivity\todo{OQESAF picture}}\label{fgr:OQESAF}
\end{figure}

In this section, we will look into the details of the comparative transport and adsorption performances of some archetypal structures (Table~\ref{table:diff}) to better understand the key factors that could explain the difference in performance. In the future, this work can be used for the design of more quantitative characteristics that could explain the better transport performance, similarly to what we did for the thermodynamic screening that led to a series of key thermodynamic descriptors that led to the design of a ML model for adsorption selectivity prediction (chapters 3--4). To achieve this, we will use a visualization tool based on the grid calculation principle shown in the dedicated section~\ref{sct:grid}, whose code is published in a Github repository: \todo{Give a github repo}.

\begin{table}[ht]
\setlength\extrarowheight{5pt}
\centering
\begin{tabular}{|l|c|c|c|c|c|c|}
\hline
  Structure &       &    &  Pore size &  Channel size   &     & Diffusion Coeff. \\
  CSD ref.\ code &  $s_0\ex{Xe/Kr}$  &  $s_1\ex{Xe/Kr}$   &    D$_{if}\ex{UFF}$ (\si{\angstrom})   &   PLD\ex{UFF} (\si{\angstrom})  &  $s\e{diff}\ex{Xe/Kr}$ &  $D\e{diff}\ex{Xe}$ (\si{\square\centi\meter\per\second}) \\
\hline
ADOGEH~\cite{Peikert_2012} & 49  &  10 & 12.9 & 5.3 & 15.5 &  5$\times$10\ex{-5}  \\
XUNSOQ~\cite{Abrahams_2014} & 38  & 48 &  5.6 & 4.8 &  0.23 &  7$\times$10\ex{-6} \\
GUMDEZ~\cite{Yin_2014} & 56 & 42 &  5.5 & 5.1 &  0.55 &  7$\times$10\ex{-5} \\
QOZDOY~\cite{Zhang_2001} & 52  & 37 &  5.6 & 5.0 &  0.45 &  7$\times$10\ex{-5} \\
MISQIQ~\cite{Tong_2013} & 140 & 37 &  4.6 & 4.5 &  1.4 &  2$\times$10\ex{-4} \\
TONBII~\cite{Du_2010} & 44 & 35 &  5.1 & 4.8 &  0.86 &  1$\times$10\ex{-4} \\
BAEDTA01_clean	37.741382	152.160042	5.69169	4.63570	0.400504	
OQESAF_clean	27.903385	27.667023	5.79096	4.96599	16.856013	0.000037
\hline
\end{tabular}
\caption{Transport and thermodynamic performances of top performing structures of CoRE MOF 2019 screened out by the approach developed in section~\ref{sct:diff_screen}. The thermodynamic properties are determined using $s_0\ex{Xe/Kr}$ and $s_1\ex{Xe/Kr}$ that correspond to the xenon/krypton adsorption selectivity values respectively at infinite dilution and ambient pressure condition. The pore size is defined as the largest cavity along a free diffusion path D$_{if}\ex{UFF}$ and the channel size is defined using the pore limiting diameter PLD\ex{UFF} using atom radii defined by the UFF. The transport properties are evaluated using the xenon/krypton diffusion selectivity $s\e{diff}\ex{Xe/Kr}$ and the xenon diffusion coefficient $D\e{diff}\ex{Xe}$ calculated by the MD-based screening presented above. }\label{table:diff}
\end{table}
\todo{capacity ?}

The structure ADOGEH\autocite{Peikert_2012} was not found when crossing the transport data with ambient-pressure selectivity values by with the infinite dilution ones, which explains that the selectivity $s_1$ is rather low ($10$) compared to the other materials (over $35$). This structure was actually detected when looking at the selectivity $s_0$ at infinite dilution, because of its extraordinary diffusion selectivity around ($10$). This would mean that as membrane material, its selectivity would be around $100$, which is on of the highest. Even as an adsorption-based separation material, it has an outstanding low-pressure selectivity of $49$ coupled with the high diffusion selectivity it can be considered for some applications at very low partial pressure of xenon and krypton. 

\begin{figure}[ht]
  \centering
    \includegraphics[width=0.48\textwidth]{figures/5-diffusion/viz/ADOGEH_Xe.jpg}
    \includegraphics[width=0.48\textwidth]{figures/5-diffusion/viz/ADOGEH_Kr.jpg}
    \caption{3D volume plot of the xenon (left) and krypton (right) interaction energy values inside the material ADOGEH\autocite{Peikert_2012} calculated using an energy grid as described in section~\ref{sct:grid}. }\label{fgr:ADOGEH}
\end{figure}

And even as an experimental material, the diffusion properties of xenon and krypton in this material are extremely interesting in themselves. If we look at the Figure~\ref{fgr:perm_selec0}, only two materials have a diffusion selectivity over $10$, but the other one actually has a falsely high value of diffusion selectivity due to the above-mentioned randomness of the initial position in the MD simulation and the presence of two types of channel (see Figure~\ref{fgr:false_diff_sele}). ADOGEH is therefore the only material that shows such a high diffusion selectivity among all materials screened for their diffusion performance. In a unidimensional system, it is actually more natural to have a higher or equivalent diffusion coefficient for krypton than for xenon due to the obvious size difference. It is possible to explain this exceptional by a special mechanism that happens in the tridimensional network of channel of ADOGEH. As we can see on the Figure~\ref{fgr:ADOGEH}, for both xenon and krypton all dimensions are available for diffusion through the channels in the three cardinal directions. However, when we look at the sort of ``pocket'' (not a blocking pocket in the MD simulation) that connects the channels diagonally, the access to it is clearly not the same if we compare the two 3D energy grid plots. For xenon, on the left, the connection is way thinner than for krypton for the same energy threshold. This can be interpreted as a higher energy barrier for xenon than for krypton to access this ``pocket''. But, why would this explain the unusual difference in diffusion coefficients for xenon and krypton? This can be explained by the fact that the directions in which a krypton can diffuse is therefore much higher, which means that it has a higher probability of turning around than xenon. To say it more explicitely, a xenon can diffuse in the 3D space by taking the 6 main channels only, but a krypton would lose time in small channels connecting the cardinal directions --- opening up dimensions decreases the diffusion coeffcient ($\langle{r(t)}^2\rangle=2d\tau$). Moreover, when a krypton takes the small channel toward the ``pocket'', it would have a non negligible residence time inside, which further slows it down compared to a xenon. These ``pockets'' can be seen as traps for krypton if it were a race between both adsorbate inside the nanoporous material. 

\todo{HERE}

\begin{figure}[ht]
  \centering
    \includegraphics[width=0.48\textwidth]{figures/5-diffusion/viz/XUNSOQ.jpg}
    \includegraphics[width=0.48\textwidth]{figures/5-diffusion/viz/GUMDEZ.jpg}
    \caption{\todo{}}\label{fgr:Other}
\end{figure}

\begin{figure}[ht]
  \centering
    \includegraphics[width=0.48\textwidth]{figures/5-diffusion/viz/BAEDTA01.jpg}
    \includegraphics[width=0.48\textwidth]{figures/5-diffusion/viz/MISQIQ.jpg}
    \caption{\todo{}}\label{fgr:drop_selec}
\end{figure}

\begin{figure}[ht]
  \centering
    \includegraphics[width=0.48\textwidth]{figures/5-diffusion/viz/TONBII.jpg}
    \caption{\todo{}}\label{fgr:TONBII}
\end{figure}

\todo{write about this \autocite{Stanton_2022}}

% Les résultats d'un screening couplant thermodynamique et cinétique permettent déjà de mettre en valeur des matériaux encore inconnus de la communauté. La question du blocage cinétique est certes traitée dans certaines études,\autocite{Stanton_2022} mais aucune approche systématique n'a encore été développée. Ces structures ont des sélectivités élevées mais ne sont pas les plus sélectives; elles auraient donc été écartées dans un screening standard basé uniquement sur la thermodynamique. 

\todo{transition to faster approaches}
% Cette approche basée sur des trajectoires de dynamique moléculaire présente toutefois ses propres défauts. En effet, la méthode est très lente (quelques jours de simulation par structure), les structures avec différents canaux non connectées posent problème car l'adsorbat ne peut diffuser que dans un seul des deux canaux. Ainsi, il faudrait dans certains cas faire plusieurs trajectoires ce qui demande encore davantage de temps de calcul. De plus, si l'on veut ajouter de la flexibilité, cette méthode ne pourrait être utilisée que sur quelques structures. C'est pourquoi, un algorithme bien plus efficace est en cours d'implémentation en C++ pour accélérer les calculs de diffusivité. Un tel algorithme est déjà conçu par le groupe de Berend Smit à l'EPFL \autocite{Mace_2019}, mais le code étant sous Matlab (langage peut efficace), il est impossible de l'utiliser dans des procédures de screening. Cet algorithme détermine des sites d'adsorption et des états de transition pour passer d'un site à un autre. Puis à l'aide de simulations de Monte Carlo cinétique l'adsorbat se déplace d'un site à un autre au rythme d'une constante cinétique déterminée par l'état de transition. Cette méthode permet également d'identifier en avance les différents canaux et on assure que chacun d'entre eux sont traversés par des molécules. Par conséquent, ce nouveau code permettrait de résoudre les problèmes de temps de calcul et de fiabilité rencontrés avec la dynamique moléculaire. 

\section{Fast diffusion calculation algorithm}\label{sct:algo_diff}

\subsection{Implementation in C++}


Grid calculation similar to the one in chapter 3

Breadth-first search

\subsection{Preliminary results}

\todo{Barrier energy}

\begin{figure}[ht]
  \centering
    \includegraphics[width=0.48\textwidth]{figures/5-diffusion/difflog_Df-ccdc_barrier.pdf}
    \includegraphics[width=0.48\textwidth]{figures/5-diffusion/difflog_Df-uff298K_barrier.pdf}
    \caption{}\label{fgr:diff_pld_barrier}
\end{figure}

Explains the difference in diffusivity values for mateirals with a PLD superior to \SI{6}{\angstrom}
\begin{figure}[ht]
  \centering
    \includegraphics[width=0.48\textwidth]{figures/5-diffusion/difflog_barrier_Df_uff.pdf}
    \includegraphics[width=0.48\textwidth]{figures/5-diffusion/difflog_barrier_Df_uff_2.pdf}
    \caption{}\label{fgr:barrier_diffusion}
\end{figure}

\subsection{Visualization tool}

\todo{Take some examples for the vizualisation with comparison to the pore size and diffusion coefficient}

\section{First ML prediction Model}

Noisy data to begin with

\textbf{Performances}

\begin{figure}[ht]
  \centering
    \includegraphics[width=0.48\textwidth]{figures/5-diffusion/diffusion_prediction.pdf}
    \caption{}\label{fgr:diffusion_pred}
\end{figure}


\textbf{Next steps}

\todo{coefficient of tortuosity ?}


\todo{find higher order relations than the diffusion for 1D frameworks. Is it possible for other D channels?}

\todo{Can be used in breakthrough simulations using RUPTURA and equations that link the diffusion coefficient to the axial dispersion coefficient a key parameter in the breakthrough modeling.}\autocite{Sharma_2023}

\todo{Broaden to the study of collective diffusion, Maxwell-stefan, Onsager, etc.}


\OnlyInSubfile{\printglobalbibliography}

\end{document}

