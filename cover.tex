%!TEX root = A-web.tex
% Nicolas Castel PhD thesis -- (C) 2023

% information for cover pages

\institute{Chimie ParisTech}
\doctoralschool{Chimie Physique et\\ Chimie Analytique de\\ Paris Centre}{388}
\specialty{Chimie Physique}
\date{XX Septembre 2023}

\jurymember{1}{NAME \textsc{SURNAME}}{TITLE, PLACE/University}{Présidente du jury}
\jurymember{2}{Sof{\'{i}}a \textsc{Calero}}{Professeure, Eindhoven University of Technology}{Rapportrice}
\jurymember{3}{Magali \textsc{Benoit}}{Directrice de Recherche, Université de Toulouse}{Rapportrice}
\jurymember{4}{Johann \textsc{William}}{Responsable modélisation et IA, Orano }{Membre invité}
\jurymember{5}{Isabelle \textsc{Hablot}}{Manager recherche et développement, Orano}{Représentante industrielle}
\jurymember{6}{Philippe \textsc{Guilbaud}}{Chef de laboratoire et Directeur de Recherche, CEA}{Directeur de thèse}
\jurymember{7}{François-Xavier \textsc{Coudert}}{Directeur de Recherche, Chimie ParisTech}{Directeur de thèse}

\frabstract{

Mes travaux portent sur la description microscopique de matériaux nanoporeux à différent niveaux de modélisation moléculaire, incluant des approches de criblage à haut débit et d’apprentissage statistique sur des bases de données de matériaux.
}

\enabstract{

My work concerns the high-throughput screening of nanoporous materials (mostly but not only metal-organic frameworks) at different levels of molecular modeling, focusing on Xe/Kr separation as an application.

}

\frkeywords{simulation moléculaire, matériaux nanoporeux, criblage haut-débit, adsorption, apprentissage statistique}
\enkeywords{molecular simulation, nanoporous materials, high-throughput screening, adsorption, machine learning}
