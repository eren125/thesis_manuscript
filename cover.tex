%!TEX root = main.tex
% Nicolas Castel PhD thesis -- (C) 2023

% information for cover pages

\institute{Chimie ParisTech}
\doctoralschool{Chimie Physique et\\ Chimie Analytique de\\ Paris Centre}{388}
\specialty{Chimie Physique}
\date{XX Septembre 2023}

\jurymember{1}{NAME \textsc{SURNAME}}{TITLE, PLACE/University}{Présidente du jury}
\jurymember{2}{Sof{\'{i}}a \textsc{Calero}}{Professeure, Eindhoven University of Technology}{Rapportrice}
\jurymember{3}{Christelle \textsc{Miqueu}}{Maître de Conférences, Université de Pau \& Pays Adour}{Rapportrice}
\jurymember{4}{Johann \textsc{William}}{Responsable modélisation et IA, Orano}{Membre invité}
% \jurymember{5}{Isabelle \textsc{Hablot}}{Manager recherche et développement, Orano}{Représentante industrielle}
\jurymember{6}{Philippe \textsc{Guilbaud}}{Chef de laboratoire et Directeur de Recherche, CEA}{Encadrant thèse}
\jurymember{7}{François-Xavier \textsc{Coudert}}{Directeur de Recherche, Chimie ParisTech}{Directeur de thèse}

\frabstract{

Cette thèse vise à améliorer la séparation des gaz xénon et krypton en utilisant des matériaux nanoporeux. je développe des outils de description microscopique de ces matériaux, en utilisant différents niveaux de modélisation moléculaire. J'intègre des approches de criblage à haut débit et d'apprentissage statistique en exploitant des bases de données de matériaux pour évaluer rapidement leurs performances. 
\todo{Pour cela thermo, transport, etc.}
Ces travaux contribuent à la recherche de solutions plus efficaces et durables pour la séparation de gaz dans diverses industries.
}

\enabstract{

This thesis aims to improve the separation of xenon and krypton gases using nanoporous materials. I am developing tools for the microscopic description of these materials, using various levels of molecular modeling. I integrate high-throughput screening approaches and statistical learning by leveraging material databases to rapidly evaluate their performance. These efforts contribute to the search for more efficient and sustainable solutions for gas separation in various industries.

}

\frkeywords{simulation moléculaire, séparation de gaz, adsorption, matériaux nanoporeux, \\\\ criblage haut-débit, apprentissage statistique}

\enkeywords{molecular simulation, gas separation, adsorption, nanoposrous materials, \\\\high-throughput screening , machine learning}
