%!TeX root = 1-screening
\documentclass[main]{subfiles}

\begin{document}

\chapter{High-throughput computational screening of nanoporous materials}
\vspace*{-1\baselineskip}


\section{Introduction to the main screening tools}

\subsection{Databases}

\todo{Ren2021}

In the past decade, large-scale computational screening studies have become an important part of the materials science innovation pipeline,\cite{Hautier_2019, Cole_2020} trying to move beyond the serendipitous model of materials discovery.\cite{Ludwig_2019, Stein_2019} High-throughput computational discovery techniques are used in the generation of novel hypothetical structures for screening,\cite{Wilmer_2011, Boyd_2016} as well as in trying to explore more in depth and more systematically the materials whose structure has already been published, in order to map their physical and chemical properties.\cite{GomezGualdron_2014,Moliner_2019,SalcedoPerez_2019} While the idea of large-scale exploration of materials is not new, and such databases --- whether experimental or computational in the source of their data --- have been around for several decades now,\cite{PDB_1971, Grazulis_2009, Groom_2016} this field has recently seen a rapid expansion enabled by several factors.

\subsection{Simulation tools}

\subsection{Machine learning assisted screening}


\section{A literature overview}

\subsection{Thermodynamic adsorption properties}

\subsubsection{Gas storage}

\subsubsection{Gas separation}

\subsection{Transport adsorption properties}

\subsubsection{Kinetic properties}

Used in breakthrough simulation

\subsubsection{Membrane materials}

\subsection{Non-adsorption properties}

\subsubsection{Catalytic activity}

\subsubsection{Mechanical properties}

\subsubsection{Thermal properties}

\section{Consequences for xenon/krypton separation}

\subsection{Status quo}

\subsubsection{What is done in Xe/Kr separation}

\subsubsection{What can be learned in the other fields}

\subsection{Future perspectives}

Main improvement points

\subsubsection{Faster energy sampling}
Integration in ml
\subsubsection{Faster diffusion estimation}
sqs
\subsubsection{Flexibility OMS}

\OnlyInSubfile{\printglobalbibliography}

\end{document}
