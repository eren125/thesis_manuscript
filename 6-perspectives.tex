%!TeX root = 6-perspectives.tex
\documentclass[main]{subfiles}

\begin{document}

\chapter{Toward the next generation of screenings}
\vspace*{-1\baselineskip}

\section{Limits of the Current Screening Methodologies}

\todo{put graph comparing experiment and UFF}

The beginning of some explication

\section{Flexibility}
Final screening step, easy integration into the workflow of current sreenings
\subsection{Problem, literature}

\subsection{Database approach}

\subsection{Perspectives: Snapshot method}

\section{Noble Gas Polarizability}

\subsection{Problem definition}

Best materials use polarization effects \autocite{Li_2019,Pei_2022}

Talk about the order of magnitude of the different interactions > charge--(induced dipole high magnitude)

Standard methods failing to describe oms\autocite{Perry_2014} 

\todo{faire référence à 2-thermo partie sur les interactions\ref{sct:interaction}}




\subsection{Studying the polarization}

Inspired by works on the subject\autocite{Lachet_1998,Becker_2017} 

\todo{essayer d'ajouter la polarisabilité pour PEI et al. et Li et al.}

Xe/Kr difference of polarisability
Open Metal Sites/polar groups
\todo{20220421\_pres}

\todo{Not the best material, but interesting discussion on open metal site effect}
Tao et al.\autocite{Tao_2020} looked at tuning (and improving) the selective adsorption of Xe over Kr by MOF open metal sites in the UTSA-74 framework structure.


\subsection{Perpectives}

\OnlyInSubfile{\printglobalbibliography}

\end{document}
