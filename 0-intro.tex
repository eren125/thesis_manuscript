%!TEX root = A-web.tex
% Emmanuel Ren PhD thesis -- (C) 2023

\chapter*{General introduction}
\pagenumbering{arabic}

% separation in general
Industrial gas separation processes is widely used to supply purified reactants and inert gases for the chemical, the health, the agricultural and the food industries. It can also be used to mitigate negative environmental impact of some industrial activities: in concrete or steel production factories, the highly problematic \ce{CO2} emissions could be separated from other atmospheric gases and captured; in nuclear treatment plants, volatile radioactive compounds (e.g., $^{85}$Kr) can also be captured through an effective separation. Different small molecules are usually considered in these processes such as nitrogen, oxygen, carbon
dioxide, hydrogen, methane, nitrous oxide or the noble gases. The xenon/krypton separation studied in this thesis is usually performed to extract xenon and krypton from the atmosphere,\autocite{kerry2007industrial} but the nuclear industry constitutes a more abundant source of noble gases.\autocite{Banerjee_2014} 

% nanoporous
In the industry, Xe/Kr separation is usually based on the cryogenic distillation of liquified atmospheric air, which requires a lot of energy, heavy infrastructures and a thorough hazard management. The hazardousness of the process resonates with the recent industrial accidents (1997) that occurred due to the reaction of non-filtered hydrocarbons with purified liquid oxygen.\autocite{distill_accident,distill_accident2} To avoid security problems and to reduce the installation and the running cost of the gas separation process, a promising technology based on the competitive adsorption on a nanoporous material is now being actively researched. Nanoporous materials are constituted by nanoscale pores that offers a large surface area on which molecules can interact and adhere on. Industrial adsorption separation usually utilizes pressure swing adsorptions (PSA) --- the pores are loaded with a given gas mixture at high pressure and then the gas is released by applying a lower pressure. If the material preferentially load one type of molecule, the composition of the released gas would have a much higher content of this molecule, hence achieving a gradual separation. In this thesis, xenon being chemically similar to krypton, the most challenging step of the purification of xenon remains the xenon/krypton separation. For instance, some prototypes involving beds of nanoporous materials have been developed for xenon/krypton separation.\autocite{Banerjee2018} 

% computation
For the process to be viable, materials need to perform even better and many studies focus on synthesizing ever more selective materials by leveraging all chemical intuitions around noble gas adsorption properties.\autocite{Chen_2014, Li_2019, Pei_2022} In order to speed the discovery process of novel materials with key properties, computational screening can identify factors explaining the performance and pre-select candidates for further experimental studies. As recently conceptualized by Lyu et al., a synergistic workflow combining computational discovery and experimental validation can push material discovery to the next stage.\autocite{Lyu_2020, Jablonka_2022} But to efficiently guide experimental discoveries, computational chemists are facing two major challenges: generating reliably more structures and evaluating them with fast and accurate models. My work will focus on the second aspect on tool development.

The number of nanoporous materials is potentially unlimited; for the metal--organic frameworks (MOFs) alone, over 90,000 structures have been synthesized\autocite{Groom_2016} and 500,000 digitally constructed\autocite{Wilmer_2012,Boyd_2016,Colon_2017}. To deal with this ever-increasing quantity of structures, researchers are developing screening strategies to efficiently identify the best materials, while building a chemical intuition on the characteristics favorable to a high separation performance. Some studies focus on a multistep screening strategy,\autocite{Wilmer_2012,Qiao_2016,Yang_2020} while others use machine learning algorithms to speed up their screening procedures.\autocite{Simon_2015} The current screening strategies are usually based on computational tools that are more adapted to single-structure studies rather than high-throughput screenings. Moreover, in the industrial process of PSA introduced earlier and other similar technologies, many variables are required to draw a complete picture of the performance: the selectivity, the working capacity, the kinetics and thermodynamics behind the regeneration of the material (i.e., unloading the pores for another cycle).\autocite{Kumar_1994} This thesis aims at answering both these challenges by designing more efficient tools for high-throughput screening of not only the most studied selectivity performance metric, but also of transport properties and others. 


\begin{center}
    \pgfornament[width=6cm,color=CTsemi]{88}
\end{center}

In this manuscript, I will start by a literature review on the screening methodologies that are applied to very different applications of nanoporous materials. It was an opportunity to explore the different techniques that are used in a variety of research fields to inspire the current work.\autocite{Ren_2022} For instance, I proposed another angle to the screening of the separation process by breaking down the selectivity metric into thermodynamic quantities such as the enthalpy, the free energy and the entropy. This study based on time-consuming calculations revealed the effects of pressure, the thermodynamic nature of the selectivity and some structure--property relationships.\autocite{Ren_2021} To further improve the selectivity screening for Xe/Kr separation, I introduced different simulation tools to evaluate the adsorption performance of a nanoporous material.\autocite{Ren_2023} This opened up new opportunities for the development of computationally cheaper and more accurate energy descriptors for the ML prediction of Xe/Kr selectivity at ambient pressure.\autocite{Ren_2023_ml} Through this work, faster and accurate evaluation tools of the Xe/Kr separation have been developed, which potentially enables the improvement of the physical description of the system. 

As previously mentioned, the gas capacity of the nanoporous materials and the transport properties inside them are also very important metrics to evaluate the industrial separation process. The gas capacity can be obtained by GCMC calculations, and alternative methodologies were not thoroughly studied in this thesis. The fifth chapter of this thesis rather focuses on the determination of transport properties and alternative methods of evaluating it. Finally, some perspectives on the physical description of the system (flexibility and polarizability) are provided in the final chapter.


\vfill
\begin{center}
    \pgfornament[width=6cm,color=CTsemi]{75}
\end{center}
\vfill\vfill